\documentclass[11pt]{tudbeamer}
\usetheme{Luebeck}


\usepackage[utf8]{inputenc}
\usepackage{gensymb}
\usepackage{default}
\usepackage{ngerman}
\usepackage{float}
\usepackage{graphicx}
\usepackage{subcaption}
\usepackage{color}

% insert frame number
\expandafter\def\expandafter\insertshorttitle\expandafter{%
      \insertshorttitle\hfill%
\insertframenumber\,/\,\inserttotalframenumber}

% Metadata
\title{Rechnerarchitektur 2016}
\subtitle{Session 3}

\author{Fredo Erxleben}

\begin{document}
  \maketitle

\begin{frame}{Organisational}

	Next Monday is a holiday.\\
	Please do 3.11 - 3.16 at home or go to the exercises on

	\begin{tabular}{lcc}
		Tuesday, 	&1st slot, &APB E023\\
		Thursday, 	&1st slot, &APB E010\\
		Friday, 		&1st slot, &APB E006\\
	\end{tabular}

\end{frame}

\begin{frame}{Today}

	\begin{itemize}
		\item Basic arithmetic operations on IEEE 754 floating point numbers\dots
		\item Basic arithmetic operations on fixed-point numbers\dots
		\item \dots and their properties and caveats
	\end{itemize} 

\end{frame}

\section{Exercise 1}
\section{Exercise 2}
\section{Exercise 3}

\begin{frame}[allowframebreaks]{Exercise 3.5}

	Addition and Subtraction:\\

	\begin{itemize}
		\item Equalize exponents (only the difference of the exponents is relevant)
		\item Add / subtract mantissas as usual
		\item Normalize if required
	\end{itemize}

\framebreak

	Multiplication and Division: \\
	$\rightarrow$ Blackboard
\end{frame}

\begin{frame}{Exercise 3.6}

	Use denormalized form:
	$$Z = (-1)^v \cdot 0.f\, \cdot \, 2^{1-B}$$

\end{frame}

\begin{frame}[allowframebreaks]{Exercise 3.7}

	Consider the addition of two n-bit numbers which results in a (n+1)-bit number:\\
	\vspace{1em}

	\begin{center}		
	\begin{tabular}{lrl}
			&		&$a_{n-1} \cdots a_0$ \\
		+	&		&$b_{n-1} \cdots b_0$ \\
		\hline 
		(+carry)	&$c_n$ 	&$c_{n-1} \cdots c_0$ \\
		\hline
		=	&$s_n$ &$s_{n-1} \cdots s_0$ \\
	\end{tabular}
	\end{center}
	
	Where $c_i$ is the incoming carry for digit $i$ and $c_{i+1}$ is the outgoing carry of digit $i$. \\
	$c_0$ is 0 if not indicated otherwise.
	
	\begin{block}{Basic idea}
		Observe the carry bits that go into and come out of the n-th digit of the result.
	\end{block}
	
\framebreak

	Depending on $a_{n-1}$, $b_{n-1}$ and $c_{n-1}$ you then can reliably determine the value of $c_n$.

	\begin{center}
	\begin{tabular}{cc|c}
		$a_{n-1}$	& $b_{n-1}$	& $c_n$ \\
		\hline
		0			& 0			& 0		\\
		0			& 1			& $c_{n-1}$ \\
		1			& 0			& $c_{n-1}$ \\
		1			& 1			& 1 \\
	\end{tabular}
	\end{center}
	
	This allows the conclusions that \dots

	\begin{block}{If $a_{n-1} \neq b_{n-1}$:}
		$c_n = c_{n-1} \Rightarrow$ incoming and outgoing carry of the last digit are equal, different sign bits of the summands generate \emph{no} overflow.
	\end{block}
	
\framebreak

	\begin{block}{If $a_{n-1} = b_{n-1} = 0$:}
		The outgoing carry $c_{n_{predicted}}$ should always be 0.\\ 
		If an incoming $c_{n-1} = 1 \neq c_{n_{predicted}}$ forces $c_n$ to become 1, an overflow has happened.
	\end{block}
	
	Example:
	\begin{center}		
	\begin{tabular}{lrl}
			&		&$01$ \\
		+	&		&$01$ \\
		\hline
		(+ carry)&	$0$	&$1-$ \\
		\hline 
		=	&$0$ 	&$10$ \\
	\end{tabular}
	\end{center}
	
\framebreak
	
	\begin{block}{If $a_{n-1} = b_{n-1} = 1$:}
		The outgoing carry $c_{n_{predicted}}$ should always be 1.\\ 
		If an incoming $c_{n-1} = 0 \neq c_{n_{predicted}}$ forces $c_n$ to become 0, an overflow has happened.
	\end{block}
	
	Example:
	\begin{center}		
	\begin{tabular}{lrl}
			&		&$10$ \\
		+	&		&$10$ \\
		\hline
		(+ carry)&	$1$	&$0-$ \\
		\hline 
		=	&$1$ 	&$00$ \\
	\end{tabular}
	\end{center}

\end{frame}

\begin{frame}{Exercise 3.8 - 3.10}
	
	3.8 $\rightarrow$ Blackboard, Homework\\
	3.9 $\rightarrow$ Homework\\
	3.10 $\rightarrow$ Blackboard

\end{frame}

\section{Wrapping up}

\begin{frame}{Last slide (finally)}

	Homework:
	\begin{itemize}
		\item Finish exercise 3.8
		\item Exercise 3.9
		\item Exercise 3.11 to 3.16
	\end{itemize}
	\vspace{1em}
	
	Next session we talk about
	\begin{itemize}
		\item Boolean logic
		\item Basic boolean circuit elements
	\end{itemize} 
	\vspace{1em}
	\textbf{Also:} Question time!

\end{frame}

\end{document}
