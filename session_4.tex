\documentclass[11pt]{tudbeamer}
\usetheme{Luebeck}


\usepackage[utf8]{inputenc}
\usepackage{gensymb}
\usepackage{default}
\usepackage{ngerman}
\usepackage{float}
\usepackage{graphicx}
\usepackage{subcaption}
\usepackage{color}

% insert frame number
\expandafter\def\expandafter\insertshorttitle\expandafter{%
      \insertshorttitle\hfill%
\insertframenumber\,/\,\inserttotalframenumber}

% Metadata
\title{Rechnerarchitektur 2016}
\subtitle{Session 4}

\author{Fredo Erxleben}

\begin{document}
  \maketitle

\begin{frame}{Today}

	Learn about:
	\begin{itemize}
		\item Boolean algebra
		\item Boolean logic circuitry
	\end{itemize}

\end{frame}

\section{Exercise 1}
\section{Exercise 2}
\section{Exercise 3}
\section{Exercise 4}

\begin{frame}[allowframebreaks]{Exercise 4.1}
	A truth table with 2 variables has $2^2 = 4$ rows. \\
	This results in $2^4 = 16$ possible result combinations.
	
\framebreak

	\begin{tabular}{lccl}
	$x_1$ 	& 1100 &&\\
	$x_0$ 	& 1010 &&\\
	\hline
			& 0000 & $f_0 = 0$ & Constant 0 \\
			& 0001 & $f_1 = x_1 x_0$ & Conjunction (AND) \\
			& 0010 & $f_2 = x_1 \bar{x_0}$ & Inhibition \\
			& \dots &&\\
	\end{tabular}	
	
\end{frame}

\begin{frame}{Exercise 4.2}

	\begin{itemize}
		\item Law of commutation
		\item Law of association
		\item Law of distribution
		\item Law of absorption
		\item Neutral element
		\item Complementary element
		\item DeMorgan's Laws
	\end{itemize}
	
	\begin{block}{Principle of duality}
		Negating each literal and switching of conjunction and disjunction yields the inverse of the original equation.
	\end{block}

\end{frame}

\begin{frame}{Exercise 4.3 \dots 4.7}

	$\rightarrow$ Blackboard

\end{frame}

\begin{frame}{Exercise 4.8}

	Two possible approaches:\\
	\begin{itemize}
		\item Algebraic simulation of the circuit (yields truth table)
		\item Symbolic calculation of the equation (yields logic formula)
	\end{itemize}
	Both solutions must yield the same result and can be transformed into one another.

\end{frame}

\section{Wrapping up}

\begin{frame}{Last slide (finally)}

	(What do we say to the god of) Homework:
	\begin{itemize}
		\item Not today
	\end{itemize}
	\vspace{1em}
	
	Next session we talk about
	\begin{itemize}
		\item More about logic circuit design
	\end{itemize} 
	\vspace{1em}
	\textbf{Also:} Question time!

\end{frame}

\end{document}
