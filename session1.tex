\documentclass[11pt]{tudbeamer}
\usetheme{Luebeck}


\usepackage[utf8]{inputenc}
\usepackage{gensymb}
\usepackage{default}
\usepackage{ngerman}
\usepackage{float}
\usepackage{graphicx}
\usepackage{subcaption}
\usepackage{color}

% insert frame number
\expandafter\def\expandafter\insertshorttitle\expandafter{%
      \insertshorttitle\hfill%
\insertframenumber\,/\,\inserttotalframenumber}

% Metadata
\title{Rechnerarchitektur 2016}
\subtitle{Session 1}

\author{Fredo Erxleben}

\begin{document}
  \maketitle

\begin{frame}{Organisational Matters}

	\begin{tabular}{ll}
		\textbf{Lecture Coordinator}	&\\
		Dr.-Ing. Thomas B. Preußer	& thomas.preusser@tu-dresden.de\\
		\\
		\textbf{Tutor}				&\\
		Bakk.-Inf. Fredo Erxleben	& fredo.erxleben@tu-dresden.de\\
		\\
		\textbf{Office}				& APB 1095\\
		\textbf{Phone}				& +49 351/ 463-38456\\
	\end{tabular}

\end{frame}

\begin{frame}{Today}

	Learn about:
	\begin{itemize}
		\item Basics concepts of representing numbers in different forms
		\item The relations between these representations
		\item How to convert numbers from one systems into another 
	\end{itemize}

\end{frame}

\section{Exercise 1}

\begin{frame}{Exercise 1.1}

	\begin{block}{Definition: Alphabet}
		An alphabet is a \textbf{finite set} of symbols.
	\end{block}
	
	According to \textit{DIN 43300} it also has a total order.
	
	\begin{block}{In math:}
		$\Sigma = \lbrace \sigma_0 \dots \sigma_n \rbrace$
		where $\sigma_0 \dots \sigma_n$ are symbols
	\end{block}
	
	Also: $|\Sigma|$ is the number of symbols in the alphabet $\Sigma$.

\end{frame}

\begin{frame}{Exercise 1.2}

	\begin{block}{Definition: Code}
		A code is the rule describing how information is represented using an alphabet.
	\end{block}
	
	c.f.: Lecture \textit{Informations- und Kodierungstheorie}
	
	\begin{block}{Definition: Code density}
		Assume a code with
		\begin{itemize}
			\item Code word length $n$
			\item Encoding $q$ different units of information
			\item Using an alphabet $\Sigma$
		\end{itemize}
		
		If $$|\Sigma|^{n-1} < q \leq |\Sigma|^n$$ is true, the code is called \textbf{dense}.
	\end{block}

\end{frame}

\begin{frame}{Exercise 1.3}

	\begin{block}{Definition: Code table}
		A Code table is the assignment of different encodings to the same information, usually in the form of a table. (Duh!)
	\end{block}
	
\end{frame}

\begin{frame}{Exercise 1.4}
	
	\begin{block}{Definition: Binary code}
		Each code with $|\Sigma| = 2$ is a binary code.
	\end{block}
	
	Often $\Sigma = \lbrace 0, 1 \rbrace$ is used.

\end{frame}

\begin{frame}{BCD code}
	\centering
	
	BCD = Binary coded decimal
	\begin{itemize}
		\item Binary code
		\item Each code word represents a decimal number
		\item In the \textbf{dense} BCD code, each code word has length 4 (\textit{tetrade})
	\end{itemize}
	
	\begin{tabular}{rc|rc}
		\textbf{Decimal}	& \textbf{Tetrade} & \textbf{Decimal} & \textbf{Tetrade} \\
		0 & 0000 & 5 & 0101 \\
		1 & 0001 & 6 & 0110 \\
		2 & 0010 & 7 & 0111 \\
		3 & 0011 & 8 & 1000 \\
		4 & 0100 & 9 & 1001 \\
	\end{tabular}
	
	The tetrades \textit{0101} to \textit{1111} are unused. (\textit{pseudo-tetrades})

\end{frame}

\begin{frame}{1-of-n code}
\centering

	\begin{itemize}
		\item Also called 1-hot code
		\item All code words have the same length
		\item Each code word contains only one \textit{1}
		\item Often used in state automata ($\rightarrow$ Exercise 7 ff.)
	\end{itemize}
	
	Let's say we have four states $Z_0 \dots Z_3$ we want to encode:
	
	\begin{tabular}{rc}
		\textbf{State} & \textbf{Encoding} \\
		$Z_0$ & 0001 \\
		$Z_1$ & 0010 \\
		$Z_2$ & 0100 \\
		$Z_3$ & 1000 \\
	\end{tabular}
	
	\textbf{Caution:} \textit{0000} is not a code word in a 1-of-n code.

\end{frame}

\begin{frame}{ASCII}

	ASCII = American Standard Code for Information Interchange (aka. ANSI X3.4-1986)
	
	\begin{itemize}
		\item 7-bit binary code
		\item Contains printable and non-printable characters
		\item Contained as sub-code in nearly any 8-bit character set
		\begin{itemize}
			\item As well as in Unicode
			\item Exception: EBCDIC
		\end{itemize}
	\end{itemize}
	
	Code table and details $\rightarrow$ https://\textbf{en}.wikipedia.org/wiki/ASCII

\end{frame}

\begin{frame}{UTF-8}

	\begin{itemize}
		\item Around since 1993, latest revision 2003 (UTF-16 compatibility)
		\item Variable code word length (8 \dots 32 bit)
		\item \textit{Prefix property} allows for error checking and stream decoding
		\begin{itemize}
			\item No code word is the prefix of another code word
		\end{itemize}
	\end{itemize}
	
	Code table and details $\rightarrow$ https://\textbf{en}.wikipedia.org/wiki/UTF-8

\end{frame}

\begin{frame}{Exercise 1.5}
\centering

	Smallest unit of information is 1 \textbf{bit} ( = Binary Digit)
	
	\begin{tabular}{ll}
		Byte 	& = 8 bit (usually\dots) \\
		Nibble 	& = $\frac{1}{2}$ byte (aka. Half-byte) \\
	\end{tabular}
	\vspace{1em}\\
		A machine \textbf{word} is the bit-size used natively by an architecture \\
		(Or programming language). \\
		Except at Intel\texttrademark \dots
	
	\begin{block}{Intel\texttrademark says}
		1 word = 16 bit, no matter the architecture
	\end{block}
	
	\textbf{Conclusion:} RTFM.

\end{frame}

\section{Exercise 2}

\begin{frame}[allowframebreaks]{Exercise 2.1}

	\begin{tabular}{lrl}
		\textbf{System} & \textbf{Base} & $\Sigma$ \\
		Dual & 2 			& $\lbrace 0,1 \rbrace$\\
		Octal & 8 			& $\lbrace 0,1,2,3,4,5,6,7 \rbrace$\\
		Decimal & 10 		& $\lbrace 0\dots9 \rbrace$\\
		Hexadecimal & 16 	& $\lbrace 0 \dots 9, A \dots F \rbrace$\\
		Sexagesimal & 60 	& lol, nope \\
	\end{tabular}

\framebreak

	\begin{block}{The simple approach}
		\begin{enumerate}
			\item Find the next smallest power of the \textit{base of the target system}.
			\item Subtract it from your current number.
			\item If you have not reached 0, repeat from step 1.
			\item Add up all the subtracted powers of the base system according to the formula on the slides
		\end{enumerate}
	\end{block}
	
	Example $\rightarrow$ blackboard

\framebreak
\centering

	Powers of 2 cheat sheet
	\vspace{1em}
	
	\begin{tabular}{rr|rr}
	x & $2^x$ & x & $2^x$ \\
	\hline
	0 & 1 & 		8 & 256 \\
	1 & 2 & 		9 & 512 \\
	2 & 4 & 		10 & 1024 \\
	3 & 8 &  	11 & 2048 \\
	4 & 16 & 	12 & 4096 \\
	5 & 32 & 	13 & 8192 \\
	6 & 64 &  	14 & 16384\\
	7 & 128 &	15 & 32768 \\
	\end{tabular}

\framebreak

	\begin{block}{Alternative approach}
		\begin{description}
			\item[\textbf{Whole part}] Iterative division (with remainder) by target base in original system
			\item[\textbf{Fractional part}] Iterative multiplication with target base in original system
		\end{description}
	\end{block}
	
	Example $\rightarrow$ blackboard

\end{frame}

\begin{frame}{Exercise 2.2}

	Formation of triades (octal) or tetrades (hexadecimal) that can be transcoded independently.\\
	Example $\rightarrow$ blackboard

\end{frame}

\begin{frame}{Exercise 2.3 and 2.4}
	\centering
	
	Your turn\dots (aka. homework)

\end{frame}

\begin{frame}[allowframebreaks]{Exercise 2.5}

	For a given value to be converted from $B_0$ to $B_1$
	let $n_i$ be the number of digits of the values representation in $B_i$.
	To retain precision, $B^{n_0}_0 = B^{n_1}_1$ must be true. Ergo:
	\begin{align}
		B^{n_0}_0 &= B^{n_1}_1 \\
		n_0 \cdot \log B_0 &= n_1 \cdot \log B_1 \\
		n_1 &= \frac{\log B0}{\log B_1} \cdot n_0
	\end{align}

\framebreak

	\centering
	A few examples
	\vspace{1em}
	
	\begin{tabular}{r|llll}
		$B_0 \setminus B_1$ & 2 & 8 & 10 & 16 \\
		\hline
		2 	& 1 			& 0,333\dots	& 0,301\dots	& 0,25 \\
		8	& 3			& 1			& 0,903\dots	& 0,75 \\
		10	& 3,322\dots	& 1,107\dots	& 1			& 0,830\dots \\
		16	& 4			& 1,333\dots	& 1,204\dots	& 1 \\
	\end{tabular}

\framebreak
	
	\begin{block}{Example: octal to decimal}
		Assume an octal number has 12 digits.
		$$\frac{\log 8}{\log 10} \cdot 12 = 0,903 \cdot 12 = 10.836$$
		It therefore needs 11 digits in decimal.
	\end{block}

\end{frame}

\section{Wrapping up}

\begin{frame}{Last slide (finally)}

	Homework:
	\begin{itemize}
		\item Exercise 2.3
		\item Exercise 2.4
		\item Exercise 2.6
		\item Learn the powers of 2\dots
	\end{itemize}
	\vspace{1em}
	
	Next session we talk about
	\begin{itemize}
		\item Negative numbers
		\item Floating point numbers
	\end{itemize} 
	\vspace{1em}
	\textbf{Also:} Question time!

\end{frame}

\end{document}
