\documentclass[11pt]{tudbeamer}
\usetheme{Luebeck}


\usepackage[utf8]{inputenc}
\usepackage{gensymb}
\usepackage{default}
\usepackage{ngerman}
\usepackage{float}
\usepackage{graphicx}
\usepackage{subcaption}
\usepackage{color}

% insert frame number
\expandafter\def\expandafter\insertshorttitle\expandafter{%
      \insertshorttitle\hfill%
\insertframenumber\,/\,\inserttotalframenumber}

% Metadata
\title{Rechnerarchitektur 2016}
\subtitle{Session 2}

\author{Fredo Erxleben}

\begin{document}
  \maketitle

\begin{frame}{Today}

	Learn about:
	\begin{itemize}
		\item Representing negative numbers
		\item Floating point numbers
	\end{itemize}

\end{frame}

\section{Exercise 1}
\section{Exercise 2}
\begin{frame}[allowframebreaks]{Exercise 2.7}

	Option 1: Sign-Value-Representation \\
	Encode the sign in a defined bit (Usually the upper one)\\
	\textbf{Problem: Two representations for $0$}
	
\framebreak

	Option 2: Defining a Bias $K$ \\
	$$Z = \left( \sum_{i=0}^{n-1} B^i \cdot z_i \right) - K$$
	
\framebreak

	Option 3: Interpret numbers with the set upper bit at position $n-1$ as negative. \\
	\begin{itemize}
		\item (B-1)-Complement
		\item B-Complement
	\end{itemize}

\framebreak

	Option 3a: (B-1)-Complement (For $B=2$: One-Complement)\\
	$$-Z = B^n-1-Z$$ \\
	Arithmetic calculations are done in the ring $Z_{B^n - 1}$ e.g. $mod \, B^n - 1$ \\
	\textbf{Problem: Two representations for $0$}
	
\framebreak

	Option 3b: B-Complement (For $B=2$: Two-Complement) \\
	$$-Z = B^n-Z$$ \\
	Arithmetic calculations are done in the ring $Z_{B^n}$ e.g. $mod \, B^n$
	
\framebreak

	A handy trick: If you already have the (B-1)-Complement you can easily calculate the b-Complement and vice versa. \\

	\begin{align}
	-Z &= B^n - Z \\
	&= (B^n-1-Z) + 1
	\end{align}
	
	For $B=2$ the term in parenthesis becomes a bitwise negative.	

\end{frame}

\begin{frame}{Exercise 2.8}

	If there is no carry into the leading digit, a $0$ becomes $B-1$ \\
	$\Rightarrow$ A positive number becomes negative 

	Example $\rightarrow$ blackboard \\
	Rest is homework.

\end{frame}

\begin{frame}{Exercise 2.9 and 2.10}

	One example $\rightarrow$ blackboard \\
	Rest is homework.

\end{frame}

\section{Exercise 3}

\begin{frame}[allowframebreaks]{Exercise 3.1}

	How to create a IEEE 754-conform floating point number:

	\begin{enumerate}
		\item Determine the sign
		\item Convert the absolute value into a dual number
		\item Normalise the converted absolute value
		\item Determine the characteristic
		\item Combine the parts into a complete number
	\end{enumerate}

\framebreak

	Example $\rightarrow$ blackboard \\
	
	\begin{block}{Caution!}
	If $c < 1$, use the denormalized form.\\
	If $c \geq 2B+1$, encode the overflow using $\pm\infty$
	\end{block}

\end{frame}

\begin{frame}[allowframebreaks]{Exercise 3.2}

	Biggest possible (absolute) value:
	
	\begin{itemize}
		\item Sign does not matter
		\item Biggest normalized characteristic $\hat{c}=254_{10}$
		\item Biggest mantissa $\hat{M} = 1.11\dots1_2$ 
	\end{itemize}
	
	\begin{align}
	\hat{Z}	&= \pm\hat{M} 					\cdot 2^{\hat{c}-B} \\
			&= \pm\left( 2 - 2^{-23}\right) 	\cdot 2^{127} \\
			&= \pm\left(2^{128} - 2^{104}\right)
	\end{align}

	Binary version $\rightarrow$ blackboard
	
\framebreak

	Smallest possible (absolute) value:
	
	\begin{itemize}
		\item Sign does not matter
		\item Smallest denormalized characteristic $\breve{c}=0$
		\item Smallest mantissa $\breve{M} = 0.00\dots01_2$ 
	\end{itemize}
	
	\begin{align}
	\breve{Z}	&= \pm\breve{M} 				\cdot 2^{1-B} \\
				&= \pm\left(2^{-23}\right) 	\cdot 2^{-126} \\
				&= \pm 2^{-149}
	\end{align}

	Binary version $\rightarrow$ blackboard

\end{frame}

\begin{frame}{Exercise 3.3 and 3.4}

	Exercise 3.3 is homework\dots \\
	The calculation parts in 3.4 you can do already, the questions require some pondering.

\end{frame}

\section{Wrapping up}

\begin{frame}{Last slide (finally)}

	Homework:
	\begin{itemize}
		\item Finish exercise 2.8, 2.9 and 2.10
		\item Exercise 3.3
		\item Prepare exercise 3.4
	\end{itemize}
	\vspace{1em}
	
	Next session we talk about
	\begin{itemize}
		\item Basic arithmetic operations on IEEE 754 floating point numbers\dots
		\item Basic arithmetic operations on fixed-point numbers\dots
		\item \dots and their properties and caveats
	\end{itemize} 
	\vspace{1em}
	\textbf{Also:} Question time!

\end{frame}

\end{document}
